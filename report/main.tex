\documentclass{article}
\usepackage[utf8]{inputenc}
\usepackage{listings}

\title{Sudoku - Low Level Programming}
\author{Khanh Duy NGUYEN TU M1 Cyber}
\date{January 2023}

\begin{document}

\maketitle
\newpage
\tableofcontents
\newpage
\section{Introduction}
In this Sudoku project, we create a software that solve and generate sudoku puzzles. Apart from traditional 9x9 sudoku, this software works on multiple size from 1x1 to 64x64.
\subsection{Terminology}
\begin{itemize}
    \item Grid : The sudoku's board
    \item Sub-grid : A set of cells in a row, column or block
    \item Candidate : Possible value in a cell
    \item Given : Filled cell
\end{itemize}
\section{Solver}
For the solver, we combine Pencil and Paper method which list all candidates for each cell then try to reduce it, and brute force which test all candidates until the right one. 
\subsection{Heuristics}
Heuristics are techniques that help narrow down candidates for each cell. All techniques introduced below act on a sub-grid.
\subsubsection{Cross Hatching}
Cross hatching is the most basic and useful technique. In a sub-grid, if a cell is given, we can then eliminate that cell's value from all other cells.
\begin{lstlisting}
function cross_hatching(subgrid)
    to_remove = empty set of value
    for each cell in subgrid
        if cell is given
            add cell's value to to_remove
    for each cell in subgrid
        if cell is not given
            remove candidates that appear in to_remove
\end{lstlisting}
\subsubsection{Lone Number}
In a sub-grid, if there is a value that only appear in one cell, that value must be set in that cell.
\begin{lstlisting}
function lone_number(subgrid)
    appeared = empty set of value
    repeated = empty set of value
    for each cell in subgrid
        add cell's candidates to appeared
        if value already exists in appeared
            add value to repeated
    lone = set of value that in appeared but not repeated
    for each cell in subgrid
        if cell is not given
            if cell has candidate in lone
                remove other candidates
\end{lstlisting}
\subsubsection{Naked Subset}
In a sub-grid, if there are N cells that have the same(or partly) N candidates, these candidates can be eliminates from other cells.
\begin{lstlisting}
function naked_subset(subgrid)
    for each cell1 in subgrid
        count = 0
        for each cell2 in subgrid
            if cell2's candidates is part or equal cell1's
                count++
        if count == number of cell1's candidates
            remove cell1's candidates from other cells
\end{lstlisting}
\subsubsection{Hidden Subset}
In a sub-grid, if there is a subset of N candidates that only appear in N cells. Then these candidates can only be in these cells, thus eliminates all other candidates in these cells.
\\
This seems hard at first glance but very simple when we look at another angle. Hidden Subset is very similar to Naked Subset to recognize. Instead of finding "a subset of N candidates only appear in N cells", we find "a subset of N cells only contain N candidates".
\\
Knowing this, at the beginning of Hidden Subset, I change sub-grid from a set of N cells, each contain list of candidates to a set of N candidates, each contain list of cells it appear in. For example: \\
Cell
\begin{itemize}
    \item 1 : 123
    \item 2 : 34
    \item 3 : 14
    \item 4 : 123
\end{itemize}
Will become: \\
Candidate
\begin{itemize}
    \item 1 : 134
    \item 2 : 14
    \item 3 : 124
    \item 4 : 23
\end{itemize}
And from this set, we apply Naked Subset logic to find the set of candidates.
\subsubsection{Heuristics Algorithm}
We first keep applying Cross Hatching and Lone Number to every sub-grid until it cannot be changed anymore. We then apply Naked Subset and Hidden Subset. If it's effective, we go back and apply Cross Hatching and Lone Number and continue this loop until all heuristics fail.
\begin{lstlisting}
    function subgrid_heuristics(subgrid, level)
        if level
            apply naked_subset, hidden_subset
        else
            apply cross hatching, lone number

    function grid_heuristics(grid)
        level = 0
        while level < 2
            for each subgrid
                subgrid_heuristics(subgrid, level)
            if nothing changes
                level++
            else
                level-- /* cap at 0 */
\end{lstlisting}
\subsection{Solver Algorithm}
We begin by checking consistency of the grid, then applying heuristics, then make a choice, create new grid by applying the choice then call solver on this new grid. If we can solve this grid, return. If not, discard the choice from the original grid then make another choice.
\begin{lstlisting}
function solver(grid)
    if grid is not consistent
        return NULL
    if grid is solved
        return grid

    make a choice
    while choice is not empty
        new_grid = copy(grid)
        apply choice to new_grid
        result = solver(new_grid)
        if result
            return result
        discard choice
        make new choice
\end{lstlisting}
\section{Generator}
\subsection{Algorithm}
Start with an empty grid, we fill the first row randomly. Then we apply the solver on this grid until it is solved. The next step is base on mode: For normal mode, we randomly remove 40\% of the values For unique mode, we do the same but in the same time applying solver after each remove to ensure that there is an unique solution for the grid, if not we remove other cell.
\begin{lstlisting}
    function generator(size, mode)
        create_grid(size);
        fill randomly first row(grid);
        grid = grid_solver(grid, mode_first);
        if mode_first
            remove 40% cells
        else
            for each step of remove 40% cells
                grid_solver(grid)
                if number of solutions != 1
                    unremove cell
\end{lstlisting}
\section{Optimization}
\subsubsection{Make choice}
Before, to make a choice, we find the cell with the least candidates. This will get expensive as the grid size get bigger since we have to search through all the cells. That's why I have decided to choose the first not empty cell.
\\
Which candidate to choose is also need to consider, in mode "--all" this doesn't seem to be a problem because we will check all the candidate anyway, but in mode first and generator, we can improve performance if we can choose the right candidate faster. That's why instead of choosing the leftmost candidate I choose randomly. In the future, we can improve this more by choosing the candidate that had appeared the least in the grid.
\subsubsection{Order of heuristics}
I apply heuristics in order that base on their complexities: 
\begin{itemize}
    \item Cross Hatching : $O(N)$
    \item Lone Number : $O(N)$
    \item Naked Subset : $O(N^{2})$
    \item Hidden Subset : $O(N^{2})$
\end{itemize}
Since Naked Subset and Hidden Subset are both expensive, We don't want to call them alot, We only call them when we are sure that Cross Hatching and Lone Number can't do anything. That's why we implement "level" into subgrid\_heuristics
\subsubsection{Good practices}
\begin{itemize}
    \item Use prefix operator (++i) instead of postfix operator (i++)
    \item If can, use bitwise operator
\end{itemize}
\end{document}
